\documentclass{article}
\usepackage[english]{babel}
\usepackage[utf8]{inputenc}
\usepackage{fancyhdr}
\usepackage{array}
\usepackage{graphicx,amsmath,amsfonts,amssymb, epsfig,color,url, amsthm}
\usepackage{siunitx}
\usepackage{setspace}
\usepackage{subcaption}
\usepackage{makecell}
\usepackage{url}
\usepackage{longtable}
\usepackage{csvsimple}
\usepackage{placeins} % put this in your pre-amble
\usepackage{flafter}
\usepackage{mathrsfs} % https://www.ctan.org/pkg/mathrsfs
\usepackage[version=4]{mhchem}
\usepackage{bm}
\usepackage{natbib}
\usepackage{mathtools}
\sisetup{detect-all}

\makeatletter
\providecommand\add@text{}
\newcommand\tagaddtext[1]{%
  \gdef\add@text{#1\gdef\add@text{}}}% 
\renewcommand\tagform@[1]{%  
  \maketag@@@{\llap{\add@text\quad}(\ignorespaces#1\unskip\@@italiccorr)}%
}
\newcommand{\unit}[1]{\ensuremath{\, \mathrm{#1}}}
\newcommand{\B}[1]{\boldsymbol{#1}}

\newcommand*\bigcdot{\mathpalette\bigcdot@{.5}}
\newcommand*\bigcdot@[2]{\mathbin{\vcenter{\hbox{\scalebox{#2}{$\m@th#1\bullet$}}}}}


\DeclarePairedDelimiter\abs{\lvert}{\rvert}%
\DeclarePairedDelimiter\norm{\lVert}{\rVert}%

\pagestyle{fancy}
\fancyhf{}
\lhead{Path Stability, Jeffrey Li}
\rfoot{Page \thepage}
\linespread{1}
\title{
  {\LARGE
   \textbf{Stability Of Paths}
   }\\~\\
   {\large
    \textbf{In what conditions is a path through a dynamic system stable?}
   }\\~\\
  {\large Mathematics Extended Essay}\\
  {\large Colonel By Secondary School}
}
\author{Jeffrey Li}
\date{2020 09 30}


\newtheorem{theorem}{Theorem}[section]
\newtheorem{lemma}[theorem]{Lemma}
\theoremstyle{definition}
\newtheorem{definition}{Definition}[section]
\newtheorem{corollary}{Corollary}[definition]

\theoremstyle{remark}
\newtheorem*{remark}{Remark}

\doublespacing

\begin{document}

\maketitle
\break

\section*{Abstract}
In this paper, we first reestablish the foundations of classical
Lyapunov point stability. We then extend Lyapunov's arguments
and establish a rigorous definition of Path Stability, in keeping 
with Lyapunov's original reasoning. Next, we note some important 
properties derived from the definition of Path Stability which 
allow the emergent behavior of the path to be decomposed and 
analyzed individually on each point. Finally, by linearizing
the differential equation describing change of state, we arrive at insights
linking the stability of the path to the Quadratic Form of
the Jacobian of the system, another problem that has extensive 
literature in the field of linear algebra, demonstrating
the concept of Path Stability has deep conceptual roots
reaching into the heart of mathematics.

\break

\tableofcontents
\break


\section{Research Question}
Given a particle travelling under a dynamic system, 
under what conditions will an arbitrarily small disturbance to the initial 
position always result in an arbitrarily small displacement to the new path,
regardless of the time? 

\section{Introduction}
In life, we can often describe the phenomenon we observe as a product of two factors:
ever-present change and an element of randomness.

Firstly, the way in which things change often can be described as a dynamic system,
which are systems whose development in time is dependent upon their current state. Dynamic systems are integral to 
many fields of study; through Hamiltonian mechanics, we can understand motion
as a dynamic system whose state is defined as the combination of position and velocity.
In ecological models, populations will grow if resources are abundant, and shrink if
resources are scarce. Hence, the change from one year to the next is largely dependent
upon the population of a species from the year before. 

Secondly, no idealized model is ever perfectly accurate to its real world counterpart
because of the prevalence of random disturbances of various magnitudes throughout
every aspect of life. The work of the Russian mathematician
Aleksandr Lyapunov \cite{Lyapunov:1992} at the turn of the 20th century has given mathematicians 
frameworks and methods for understanding the stability of a system; 
in other words, to what degree the system is  
resilient to small initial disturbances. In the following century, this
topic has been extensively studied with respect to stable \textit{points},
but not much work exists in comparison in studying the
stability of entire \textit{paths}, studying how the state of a system
develops over an arbitrarily large duration of time.


\section{Dynamic Systems}

For me, the consideration of the problem of stability arose from
contemplating a ball in a rounded crater. The ball may start at any
point along within the semi-circular cavity, but would always roll
down and come to rest at the center.  
In this paper, we will be considering systems whose state in its entirety 
can be specified using a combination of $n$ real numbers. 

\section{Lyapunov Stability of the First Kind}



We begin this section by defining the system $\mathcal{S}$ as $f_s(\B{x}) : \mathcal{D} \to \mathbb{R}^n$, where
$\mathcal{D} \subseteq \mathbb{R}^n$. We use $\B{x}_0$ to represent a general vector in $\mathcal{D}$, and define 
$P_{x_0}(t) : \left[0, \infty \right) \to \mathcal{D}$ to be a path function with the following properties.
$P_{x_0}(0) = \B{x}_0$ and $\frac{d}{dt} \left(P_{x_0}(t)\right)  = \dot{P}_{x_0}(t) = f_s(P_{x_0}(t)).$

\begin{definition}[Lyapunov Stability]
  \label{Lyapunov Stability}
  From Aleksandr Lyapunov \cite{Lyapunov:1992}, we say that the system $\mathcal{S}$ is stable at
  $\B{x}_e$ if, for every $\epsilon > 0$, there exists $\delta > 0$
  such that for every $P_{x_0}(t)$ where $\norm{\B{x}_e - P_{x_0}(0)} < \delta $,
  we have $\norm{\B{x}_e - P_{x_0}(t)} < \epsilon$
  for all $t \in \left[0, \infty \right)$.
\end{definition}

\begin{remark}
  \label{OpenBall}
  Intuitively, a system is considered stable at the point $\B{x}_e$ if, for every open ball $B_\epsilon$
  around $\B{x}_e$ of size $\epsilon$, there exists another open ball $B_\delta$ of size $\delta$ such
  that all paths that start in $B_\delta$ will stay within $B_\epsilon$.
\end{remark}

% \subsection{Unbounded Path Stability}
% Next, we will consider the unbounded stability of the path
% $P_{x_0} \in {\mathcal {D}}\subseteq \mathbb {R} ^{n} $.

% \begin{definition}[Unbounded Path Stability]
%   \label{Unbounded Stability}
%   A path $P_{x_0}$ is defined to be Unbounded Stable if, for some constant $\epsilon > 0$,
%   there exists $\delta > 0$ such that for every $\B{x}_\delta$ where
%   $\norm{\B{x}_\delta} < \delta$, then $\norm{P_{x_0}(t) - P_{\left(x_0 + x_\delta\right)}(t)} < \epsilon$ $\forall t \in \left[0, \infty \right)$
% \end{definition}

% \begin{theorem}
% Let $\B{x}_e$ be a stable point in $\mathcal{D}$. Then for every $\epsilon > 0$,
% there exists $\delta > 0$ such that all paths $P_{x_0}$ where $\norm{\B{x}_0 - \B{x}_e} < \delta$ are
% Unbounded Stable as defined in \ref{Unbounded Stability}.
% \end{theorem}

% \begin{proof}
%   Given $\epsilon$, let $B_\alpha$ be equal to the open ball centered at $\B{x}_e$ of radius $\epsilon/2$.
%   From theorem \ref{Lyapunov Stability}, there exists the open ball $B_\beta$ with radius $\beta$ such that all paths
%   starting in $B_\beta$ remain in $B_\alpha$. By the definition of an open set, each point $\B{x}_k$
%   within an open ball $B$ is itself
%   the center of an open ball of radius $r_k > 0$ contained entirely within $B$. Therefore, for some
%   $\B{x}_0 \in B_\beta$, if $\delta_2 = \beta - \norm{\B{x}_0} > \norm{\B{x}_\beta}$, then
%   $\B{x}_0 + \B{x}_\beta \in B_\beta$ for any $\B{x}_\beta$. Therefore, $\forall t \geq 0$,
%   $P_{x_0}(t) \text{ and } P_{x_0 + x_\beta}(t) \in B_\alpha$. For any two points $\B{x}_1, \B{x}_2 \in B_\alpha$,
%   $\norm{\B{x}_1 - \B{x}_2} < 2\norm{B_\alpha} = 2 \epsilon/2 = \epsilon$. Therefore
%   $\norm{P_{x_0}(t) - P_{x_0 + x_\delta}(t)} < \epsilon$ $\forall t \in \left[0, \infty \right)$.
% \end{proof}

\section{Bounded Path Stability}
Next, I used the method of defining stability in definition \ref{Lyapunov Stability}
and changed it to apply to paths as a generalization of point stability.
\begin{definition}[Bounded Stability]
  \label{Bounded Stability}
  A path function $P_{x_0}(t)$ is said to be bounded stable if, for \textit{every}
  $\epsilon > 0$, there exists $\delta > 0$ such that for every $\B{x}_\delta$ where
  $\norm{\B{x}_\delta} < \delta$, we have $\norm{P_{x_0}(t) - P_{\left(x_0 + x_\delta\right)}(t)} < \epsilon$ for all $t \in \left[0, \infty \right)$
\end{definition}

\begin{definition}[Local Attractiveness]
  \label{Local Path Attractiveness}
  A point $\B{x}_0 \in \mathcal{D}$ is said to be locally attractive if,
  for every $\epsilon > 0$, there exists $\delta \in \left(0, \epsilon\right]$ such that
  for every $\B{x}_\delta$ where $\norm{\B{x}_\delta} < \delta$,
  there exists $\sigma > 0$ for which the following is true
  for all $\Delta t \in \left[0, \sigma \right)$.
  \begin{align}
  \norm{P_{x_0}(0) - P_{\left(x_0 + x_\delta\right)}(0)} \ge \norm{P_{x_0}(\Delta t) - P_{\left(x_0 + x_\delta\right)}(\Delta t)}.
  \end{align}  
\end{definition}

\begin{theorem}[Time Invariance]
  \label{Time Invariance}
  For any two paths $P_{\B{x}_0}$ and $P_{\B{x}_1}$, if there exists $t_0, t_1 \geq 0$ such that
  $P_{\B{x}_0}(t_0) = P_{\B{x}_1}(t_1)$, then $P_{\B{x}_0}(t_0 + s) = P_{\B{x}_1}(t_1 + s)$ for all $s \geq 0$.
\end{theorem}

\begin{proof}
  From the Taylor series expansion of a function $f(t)$ as found in Taylor \cite{AdvancedCalculus},
  \begin{align}
    P_{\B{x}_0}(t_0 + s) &= P_{\B{x}_0}(t_0) + \sum_{n=1}^{\infty} \frac{s^n}{n!} P_{\B{x}_0}^{(n)}(t_0) \nonumber\\
    & = P_{\B{x}_0}(t_0) + \sum_{n=1}^{\infty} \frac{s^n}{n!} \left( \left. \frac{d}{dt}P_{\B{x}_0}^{(n-1)}(t) \right\rvert_{t=t_0} \right) \nonumber\\
    & = P_{\B{x}_0}(t_0) + \sum_{n=1}^{\infty} \frac{s^n}{n!} \left( \B{f}_s^{(n-1)} \left( P_{\B{x}_0}(t_0) \right) \right) \nonumber\\
    & = P_{\B{x}_1}(t_1) + \sum_{n=1}^{\infty} \frac{s^n}{n!} \left( \B{f}_s^{(n-1)} \left( P_{\B{x}_1}(t_1) \right) \right) \nonumber\\
    & = P_{\B{x}_1}(t_1 + s) \nonumber
  \end{align}
\end{proof}

\begin{theorem}
  \label{Attractiveness Stability}
  For some path $P_{\B{x}_0}$, if every point on the path denoted by
  $\B{x}_t = P_{\B{x}_0}(t)$, where $t \in \left[0, \infty \right)$, is locally attractive,
  then the path $P_{\B{x}_0}$ is bounded stable.
\end{theorem}

% ALSO NEED TO INCLUDE TIME INVARIANCE FML
\begin{proof}
  Let $\tau > 0$ be an arbitrarily large length of time.
  For a given $\epsilon > 0$ and some $t \in [0, \tau]$, let $\delta_t \in \left(0,\epsilon\right]$ be equal to the
  corresponding $\delta$ value given in definition \ref{Attractiveness Stability}.
  We define $\delta_{min} = min\{ \delta_t: 0 \leq t \leq \tau\}$, and let $P_{\B{x}_\phi}(t)$ be any path such that $\norm{P_{\B{x}_\phi}(0)-P_{\B{x}_0}(0)} < \delta_{min}$.

  In the following, we use proof by contradiction to prove that, for all $\B{x}_\phi$ where $\norm{\B{x}_\phi - \B{x}_0} < \delta_{min}$ and $t \in [0, \tau]$, $\norm{P_{\B{x}_\phi}(t)-P_{\B{x}_0}(t)} \leq \norm{\B{x}_\phi - \B{x}_0}$.

  We let $D(t) = \norm{P_{\B{x}_\phi}(t)-P_{\B{x}_0}(t)}$, and assume that there exists
  some $t_0 \in [0, \tau]$ such that $D(t_0) > D(0) = \norm{\B{x}_\phi - \B{x}_0}$.
  By definition of a path function, $P_{\B{x}_\phi}(t)$ and $P_{\B{x}_0}(t)$ are continuous on domain
  $\left[0, \infty\right)$ and the norm $\norm{\B{x}}$ is continuous on domain $\mathcal{D}$.
  Therefore, $D(t)$ is also continuous on $t \in \left[0, \infty\right)$, and there must exist some $t_1 \in \left[0, t_0\right)$
  such that $D(t_1) = D(0)$ and,
  \begin{align}
  D(t_k) > D(t_1) = D(0) = \norm{\B{x}_\phi - \B{x}_0}, ~~~~~~~ &\forall ~t_k \in \left(t_1, t_0\right).  \label{contradiction}
  \end{align}

  We let $\B{x}_\alpha = P_{\B{x}_0}(t_1)$ and $\B{x}_\beta = P_{\B{x}_\phi}(t_1)$. Then we have,
  \begin{align}
  D(t_1) &= \norm{P_{\B{x}_\phi}(t_1)-P_{\B{x}_0}(t_1)} \nonumber\\
  &= \norm{\B{x}_\beta - \B{x}_\alpha} \nonumber\\
  &= D(0) \nonumber\\
  & = \norm{P_{\B{x}_\phi}(0)-P_{\B{x}_0}(0)} \nonumber\\
  &= \norm{\B{x}_\phi - \B{x}_0} \nonumber\\
  &< \delta_{min} \leq \delta_{t_1},
  \end{align}
  from which there exists some $\sigma$ such that for all $\Delta t \in \left[0, \sigma\right)$, we have,
  \begin{align}
  \norm{P_{\B{x}_\alpha}(0) - P_{\B{x}_\beta}(0)} \ge \norm{P_{\B{x}_\alpha}(\Delta t) - P_{\B{x}_{\beta}}(\Delta t)} \label{attractiveness}
  \end{align}
  by local attractiveness.
      
  Since $P_{\B{x}_\alpha}(0) = \B{x}_\alpha$, we have $P_{\B{x}_0}(t_1 + s) = P_{\B{x}_\alpha}(s)$   for all $s \geq 0$ by theorem \ref{Time Invariance}. Similarly, we have $P_{\B{x}_\phi}(t_1 + s) = P_{\B{x}_\beta}(s)$ for all $s \geq 0$. By setting $s=0$ on the left side 
  of equation \ref{attractiveness} and $s=\Delta t$ on the right side of equation \ref{attractiveness}, we obtain,
  $$
  D(t_1) = \norm{P_{\B{x}_0}(t_1) - P_{\B{x}_\phi}(t_1)} \geq \norm{P_{\B{x}_0}(t_1 + \Delta t) - P_{\B{x}_\phi}(t_1 + \Delta t)},
  $$
  for some $\Delta t \in \left(0, t_0 - t_1\right)$. This is contradictory with equation~\ref{contradiction}.  
  Therefore, $\norm{P_{\B{x}_\phi}(t)-P_{\B{x}_0}(t)} \leq \norm{\B{x}_\phi - \B{x}_0} < \delta_{min}$ for all
  $t \in [0, \tau]$. Since $\tau$ can be an arbitrarily large length of time, we can conclude that $P_{\B{x}_0}(t)$ is bounded stable by denoting
  $\delta=\delta_{min}$ and $\B{x}_\delta=\B{x}_\phi - \B{x}_0$.
\end{proof}

\begin{theorem}
  \label{Local Dot Product}
  For some point $\B{x}_0$, if there exists $\delta > 0$ such that for every
  $\B{x}_\alpha$ where $\norm{\B{x}_0 - \B{x}_\alpha} < \delta$, the following inequality holds.
  $$(\B{x}_0 - \B{x}_\alpha) \bigcdot~ \left(\dot{P}_{\B{x}_0}(0) - \dot{P}_{\B{x}_\alpha}(0)\right) < 0.$$
  Then $\B{x}_0$ is locally attractive.
\end{theorem}

\begin{proof}
  Let $\B{x}(t): \mathbb{R} \to \mathbb{R}^n$ be a continuous and differentiable vector valued
  function. Let the function $d(t) = \norm{\B{x}(t)}$ be the norm of $\B{x}(t)$ for $t \in \mathbb{R}$.
  For any time $t$, we re-write $\B{x}(t)$ as $\left( x_1(t), x_2(t)..., x_n(t) \right)$
  where $x_i(t) : \mathbb{R} \to \mathbb{R}$ for $i = 1, 2, \cdots, n$. By definition of the norm in $\mathcal{R}^n$,
  \begin{align}
    (d(t))^2 &= \sum_{i=1}^n x_i(t)^2 & \Longrightarrow \nonumber\\
    \frac{d}{dt} \left((d(t))^2\right) &= \frac{d}{dt} \left(\sum_{i=1}^n x_i(t)^2\right)  & \Longrightarrow \nonumber\\
    2d(t)\dot{d}(t) &= 2\sum_{i=1}^n x_i(t)\dot{x}_i(t)  & \Longrightarrow \nonumber\\
    d(t)\dot{d}(t) &=  \B{x}(t)  \bigcdot~ \dot{\B{x}}(t) \label{Distance Derivative}
  \end{align}

  Given the paths $P_{\B{x}_0}, P_{\B{x}_\alpha}$ as defined in theorem \ref{Local Dot Product},
  we define $\B{x}_\delta(t) = P_{\B{x}_\alpha}(t) - P_{\B{x}_0}(t)$, which implies
  $\dot{\B{x}}_\delta(t) = \dot{P}_{\B{x}_\alpha}(t) - \dot{P}_{\B{x}_0}(t)$.
  Therefore,
  \begin{align}
  \B{x}_\delta(t) \bigcdot~  \dot{\B{x}}_\delta(t) & = \left(P_{\B{x}_\alpha}(t) - P_{\B{x}_0}(t)\right) \bigcdot~ \left( \dot{P}_{\B{x}_\alpha}(t) - \dot{P}_{\B{x}_0}(t)\right).\label{Distance-subtract}
  \end{align}
  Note that while $\B{x}_\delta(t)$ is a continuous and differentiable vector valued function because it is the sum
  of two continuous and differentiable path functions, $\B{x}_\delta(t)$ itself is not necessarily a valid path.

  From the inequality of theorem~\ref{Local Dot Product}, equation~\ref{Distance Derivative}, and equation~\ref{Distance-subtract},
  we have $d_\delta(0)\dot{d}_\delta(0) < 0$ where $d_\delta(t) = \norm{\B{x}_\delta(t)} = \norm{P_{\B{x}_\alpha}(t) - P_{\B{x}_0}(t)}$
  for all $t \in \mathbb{R}$. Since $d_\delta(t) \geq 0$ for all $t \in \mathbb{R}$ by the definition of the norm,
  $\dot{d}_\delta(0) < 0$. By the Mean Value Theorem \cite{AdvancedCalculus}
  and the definition of the derivative, there exists $t_1 > t_0=0$ such that for the following is true for all 
  $\Delta t \in \left(0, t_1\right)$.
  \begin{align}
  \norm{P_{\B{x}_\alpha}(\Delta t) - P_{\B{x}_0}(\Delta t)} = d_\delta(\Delta t) < d_\delta(0) = \norm{P_{\B{x}_\alpha}(0) - P_{\B{x}_0}(0)}.\label{attractive-condition}
  \end{align}
    
  Therefore, for every $\epsilon > 0$, there exists $\hat{\delta} = \min\{\epsilon, \delta\} \in \left(0, \epsilon\right]$ such that
  for all $\B{x}_\delta = \B{x}_\alpha - \B{x}_0$ with $\norm{\B{x}_\delta} < \hat{\delta}$,
  we always can find $\sigma = t_1 > 0$ for which the following is true for all $\Delta t \in \left(0, \sigma\right)$.  
  \begin{align}
  \norm{P_{\B{x}_0}(0) - P_{\B{x}_\alpha}(0)} \geq \norm{P_{\B{x}_0}(\Delta t) - P_{\B{x}_\alpha}(\Delta t)}.
  \end{align}
  By definition~\ref{Local Path Attractiveness}, the point $\B{x}_0$ is locally attractive.
\end{proof}

\section{Linearization}
To linearize the function $\B{f}_s$, we evaluate its Jacobian as defined by Taylor \cite{AdvancedCalculus}.
\begin{definition}[Jacobian]
  \label{Jacobian Approximation}
  Given $\B{x}_0$ and some unit vector $\B{x}_\delta$ for which $\norm{\B{x}_\delta} = 1$,
  the Jacobian matrix of $\B{f}_s$ evaluated at point $\B{x}_0$ is the linear transformation 
  from $\mathbb{R}^n \to \mathbb{R}^n$
  denoted by $\B{J}_{\B{f}_s}(\B{x}_0)$ and given in the following equation.
  \begin{align}
  \lim_{c \to 0} \frac{f_s(\B{x}_0 + c \B{x}_\delta) - f_s(\B{x}_0)}{c} = \B{J}_{\B{f}_s}(\B{x}_0)(\B{x}_\delta).\label{jacobian}
  \end{align}
\end{definition}
\begin{definition}[Negative Definite]
  \label{Negative Definite}
  A linear transformation denoted by \\
  $\mathscr{T}(\B{x}) \in \mathcal{L}_{n,n}$ is said to be Negative Definite \cite{LinearAlgebra}
  if, for each non-zero vector $\B{x} \in \mathbb{R}^n$, $\B{x}^T \mathscr{T}\B{x} < 0$.
\end{definition}
\begin{theorem}
  \label{Negative Definite Path Stability}
  For a given path $P_{\B{x}_0}(t)$, if for each point $\B{x}_t = P_{\B{x}_0}(t)$,
  the linear transformation led by $\B{J}_{\B{f}_s}(\B{x}_t)$
  is negative definite, then $P_{\B{x}_0}(t)$ is bounded stable.
\end{theorem}

\begin{proof}
  \label{final}
  From definition~\ref{Negative Definite}, the linear transformation led by $\B{J}_{\B{f}_s}(\B{x}_t)$
  is negative definite, if and only if for unit displacement vectors $\B{x}_\delta$, we have,
  \begin{align}
  \B(\B{x}_\delta)^T\B{J}_{\B{f}_s}(\B{x}_t)(\B{x}_\delta) =\B(\B{x}_\delta)  \bigcdot~ \left(\B{J}_{\B{f}_s}(\B{x}_t)(\B{x}_\delta)\right) < 0. \label{eq:jacobian-dot-product}
  \end{align}
  
  The inner product $\B{x}  \bigcdot~ \B{y}$ is defined to be continuous for any $\B{x}, \B{y} \in \mathbb{R}^n$.
  If $\B{x}  \bigcdot~ \B{y} < 0$, by definition of continuity, there exists some $\epsilon_1 > 0$ such that for all $\B{z} \in \mathbb{R}^n$,
  \begin{align}
    \norm{\B{z} - \B{y}} < \epsilon_1 \implies \B{x}  \bigcdot~ \B{z} < 0.  \label{eq:1}
  \end{align}
  
  From equation~\ref{jacobian}, we know that for any $\epsilon_1 > 0$, we can find $\delta_1 > 0$
  such that for scalar $c$,
  \begin{align}
    \abs{c} \in (0, \delta_1) \implies \norm{\frac{f_s(\B{x}_t + c \B{x}_\delta) - f_s(\B{x}_t)}{c}} < \epsilon_1 \label{eq:2}
  \end{align}
  
  Combining equation~\refeq{eq:1} and equation~\refeq{eq:2} into equation~\refeq{eq:jacobian-dot-product}, we obtain,
  \begin{align}
    c\B{x}_\delta  \bigcdot~ \left(\frac{f_s(\B{x}_t + c \B{x}_\delta) - f_s(\B{x}_t)}{c}\right) < 0,\nonumber\\
  \end{align}
  which gives rise to,
  \begin{align}
    \B{x}_\delta  \bigcdot~ \left(f_s(\B{x}_t + c \B{x}_\delta) - f_s(\B{x}_t)\right) < 0.
  \end{align}
    
  Because for all $\epsilon_1 > 0$, we can find $\delta_1 > 0$ such that, for all $\B{x}_{\delta_1} = |c|\B{x}_{\delta}$   
  satisfying $\norm{\B{x}_{\delta_1}} < \delta_1$, the following is true.
  \begin{align}
    \B{x}_{\delta_1}  \bigcdot~ \left(f_s(\B{x}_t + \B{x}_{\delta_1}) - f_s(\B{x}_t)\right) < 0.
  \end{align}
  By theorem \ref{Local Dot Product}, the point $\B{x}_t$ is locally attractive. Because this applies to an arbitrary $\B{x}_t = P_{\B{x}_0}(t)$,
  all points on the path are locally attractive. By theorem \ref{Attractiveness Stability},
  the path $P_{\B{x}_0}(t)$ is bounded stable as defined in \ref{Bounded Stability}.\\  
  QUOD ERAT DEMONSTRANDUM.
\end{proof}

\section{Conclusion}
This paper has explored the idea of stability and its standard mathematical 
representation. Taking Lyapunov's original idea, we have used the same
epsilon-delta formalization to expand the concept to include entire paths.
In determining if a path meets this definition of stability, it has been
shown that if the set of all points on the path meet the criteria for 
local attractiveness, as defined in the paper, then the path overall must 
also be stable. In addition, by showing that points where the quadratic form
of the Jacobian is negative determinant are always locally attractive, the paper
presents a direct method of calculating the overall stability of a path, and 
provides areas where this concept can be developed even further, most namely into
the complex vector space, as well as into stochastic or time-variant systems. 


\pagebreak
\medskip

\addcontentsline{toc}{section}{References}
\bibliographystyle{plain}
\bibliography{references}

\end{document}